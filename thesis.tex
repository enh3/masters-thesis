%% Pour voir les accents de ce fichier, assurez-vous que votre
%% éditeur de texte lise le fichier en utf-8!

%% La classe <dms> est construite au-dessus de <amsbook>, donc
%% <amsmath>, <amsfonts> et <amsthm> sont automatiquement chargés.
%% Pour un mémoire
\documentclass[12pt,twoside,maitrise]{dms}
%% Pour une thèse
%%\documentclass[12pt,twoside,phd]{dms}

\usepackage[utf8]{inputenc} %Obligatoires
\usepackage[T1]{fontenc}    %
\usepackage{epigraph}
\usepackage{csquotes}
\usepackage{graphicx}
\graphicspath{{./graphics/}}
\usepackage{caption}
%%\usepackage{chngcntr}
%%\counterwithout{figure}{chapter}
\usepackage{float}
\usepackage{listings}
%%\usepackage{courier}
\lstset{
    basicstyle=\fontsize{9}{9}\selectfont\ttfamily,
    showstringspaces=false
}

\DeclareCaptionFormat{custom}{%
    \setlength{\parindent}{0pt}% Remove the paragraph indentation
    %%\centered% Justify the caption text to the left
    \textbf{\fontsize{9pt}{0pt}\selectfont #1 #2}{\fontsize{9pt}{0pt}\selectfont #3}% Adjust the label formatting
}
\captionsetup[figure]{format=custom, labelsep=colon, belowskip=9pt, skip=9pt}

%% <lmodern> incorpore les fontes en T1, pour
%% faciliter le dépôt final. Ceci n'est pas la
%% seule option :
%%  1. Si cm-super est installé, vous pouvez enlever <lmodern>
%%     (à ce moment, la police est un peu plus fidèle
%%      au Computer Modern orginal);
%%  2. Si vous avez une police préférée, par exemple,
%%     <times> ou <euler> ou <mathpazo> (et bien d'autres),
%%     alors vous pouvez remplacer <lmodern> ci-bas.
%% Par contre, si vous faîtes face à un problème d'encapsulation
%% lors dépôt final, il se peut que la solution soit d'utiliser <lmodern>.
%% (Parfois le problème est au niveau de l'installation, donc
%%  essayez de compiler sur un autre ordinateur sur lequel vous êtes
%%  certain·e que l'installation est bonne.)
\usepackage{mathptmx}

%% Il n'est pas nécessaire d'utiliser <babel>, car
%% les commandes intégrées par la classe <dms>
%% \francais et \anglais font le travail. Néanmoins,
%% certains autres packages nécessitent <babel> (comme
%% <natbib>), donc simplement enlever les % devant <babel>
%% dans ce cas. Attention! Certains packages sont sensibles
%% à l'ordre dans lequel ils sont chargés.
%%\francais % or
%%\anglais
%%
\usepackage[english]{babel}

 % ENGLISH OPTION
 % If you call \anglais here before the \begin{document},
 % all the chater's header will be in english, even if you
 % call \francais. To change this, use
 % \entetedynamique

%% La commande \sloppy peut avoir des effets étranges sur les
%% lignes de certains paragraphes.  Dans ce cas, essayez \fussy
%% qui suppresse les effets de \sloppy.
%% (\fussy est normalement le comportement par défaut.)
%% On redéfinit \sloppy, pour tenter de réduire les comportements
%% étranges. Le seul changement apporté à la version originale
%% est la valeur de \tolerance.
\def\sloppy{%
  \tolerance 500%  %9999 dans LaTeX ordinaire, mauvaise idée.
  \emergencystretch 3em%
  \hfuzz .5pt
  \vfuzz\hfuzz}
\sloppy   %appel de \sloppy pour le document
%%\fussy  %ou \fussy

%% Packages utiles.
\usepackage{graphicx,amssymb,subfigure,icomma}
%% icomma       permet d'écrire les nombres décimaux en
%%                  français (p.ex. 1,23 plutôt que 1.23)
%% subfigure    simplifie l'inclusion de figures côtes-à-côtes

%% Packages parfois utiles.
%%\usepackage{dsfont,mathrsfs,color,url,verbatim,booktabs}
%% dsfont       symboles mathématiques \mathds
%% mathrsfs     plus de symboles mathématiques \mathscr
%% color        pour utiliser des couleurs (comparer avec <xcolor>)
%% url          permet l'écriture d'url
%% verbatim     pour écrire du code ou du texte tel quel
%% booktabs     plus de macros pour faire les tableaux
%%                  (voir documentation du package)

%% pour que la largeur de la légende des figures soit = \textwidth
%%\usepackage[labelfont=bf, width=\linewidth]{caption}

%% les 3 lignes suivante servent à l'affichage de l'index
%% dans le visionneur de pdf. <hyperref> et <bookmark>
%% devraient être les dernier package a être chargé,
%% donc chargez vos packages avant.
\usepackage{hyperref}  % Ajoute les hyperlien
\hypersetup{colorlinks=true,allcolors=black}
\usepackage{hypcap}   % Corrige la position du lien pour les images
\usepackage{bookmark} % Remédie à des petits problème
                      % de <hyperref> (important qu'il
                      % apparaisse APRÈS <hyperref>)

  % Enlever les commentaires du prochaine \hypersetup et
  % le remplir avec l'information pertinente.
  % Ceci ajoute des « méta-données » au pdf.  C'est optionnel,
  % mais recommandé. Vous pouvez voir ces méta-données en
  % ouvrant un visionneur de pdf et en cherchant les propriétés
  % du pdf. (Vous pouvez aussi tapez ' pdfinfo <nom-du-pdf> '
  % dans un terminal.) Ces données sont utiles, par exemple,
  % pour augmenter les chances qu'un algorithme de recherche
  % trouve votre document sur Internet, une fois diffusé.
%%\hypersetup{
%%  pdftitle = {Titre de la thèse / du mémoire},
%%  pdfauthor = {auteur·e},
%%  pdfsubject = {Ex: Transformation de Fourier ; régressions linéaires ; ... },
%%  pdfkeywords = {Ex: mathématiques, statistiques, groupes, variables aléatoires,...}
%%}

%% Définition des environnements utiles pour un mémoire scientifique.
%% La numérotation est laissée à la discrétion de l'auteur·e. L'exemple
%% illustré ici produit « Définition x.y.z »
%%   x = no. chapitre
%%   y = no. section
%%   z = no. définition
%% et la numérotation des corollaires, définitions, etc. se fait
%% successivement.
%%
%% Les macros \<type>name sont telles qu'ils suivent
%% la langue actuelle. (P.ex. si \francais est utilisé,
%% alors \begin{theo} va faire un Théorème et si \anglais
%% est utilisé, \begin{theo} fera un Theorem.)
%%
\newtheorem{cor}{\corollaryname}[section]
\newtheorem{deff}[cor]{\definitionname}
\newtheorem{ex}[cor]{\examplename}
\newtheorem{lem}[cor]{\lemmaname}
\newtheorem{prop}[cor]{Proposition}
\newtheorem{rem}[cor]{\remarkname}
\newtheorem{theo}[cor]{\theoremname}
\theoremstyle{definition}
\newtheorem{algo}[cor]{\algoname}
%% NOTE : Il peut être commode de redéfinir \the<type> pour
%% obtenir la numérotation désirée. Par exemple, pour
%% que les corollaires soit numérotés #section.#sous-section.#sous-sous-section.#paragraphe.#corollaire,
%% on fait
%% \renewcommand\thecor{\theparagraph.\arabic{cor}}

%%%
%%% Si vous préférez que les corollaires, définitions, théorèmes,
%%% etc. soient numérotés séparément, utilisez plutôt un bloc de
%%% commandes de la forme :
%%%

%%\newtheorem{cor}{\corollaryname}[section]
%%\newtheorem{deff}{\definitionname}[section]
%%\newtheorem{ex}{\examplename}[section]
%%\newtheorem{lem}{\lemmaname}[section]
%%\newtheorem{prop}{Proposition}[section]
%%\newtheorem{rem}{\remarkname}[section]
%%\newtheorem{theo}{\theoremname}[section]

%%
%% Numérotation des équations par section
%% et des  tableaux et figures par chapitre.
%% Ceci peut être modifié selon les préférences de l'utilisateur.
%%\numberwithin{equation}{section}
%%\numberwithin{table}{chapter}
%%\numberwithin{figure}{chapter}

%%
%% Si on veut faire un index, il faut décommenter la ligne
%% suivante. Ajouter des mots à l'index avec la commande \index{mot cle} au
%% fur et à mesure dans le texte.  Compiler, puis taper la commande
%% makeindex pour creer les indexs.  Après une nouvelle compilation,
%% vous aurez votre index.
%%

%%\makeindex

%% Il est obligatoire d'écrire à double interligne
%% ou à interligne et demi. On peut soit utiliser
%% le package <setspace> ou \baselinestretch.
%% Le package est un peu plus propre, mais le choix
%% reste à la discrétion de l'usager.
\usepackage[onehalfspacing]{setspace}
 % ou
%%\renewcommand{\baselinestretch}{1.5}

%%%%%%%%%%%%%%%%%%%%%%%%%%%%%%%%%%%%%%%%%%%%%%%%%%%%%%%%%%%%
%%%%%%%%%%%%%%%%%%%%%%%%%%%%%%%%%%%%%%%%%%%%%%%%%%%%%%%%%%%%
%%%%%%%%%%                                     %%%%%%%%%%%%%
%%%%%%%%%% D é b u t    d u    d o c u m e n t %%%%%%%%%%%%%
%%%%%%%%%%                                     %%%%%%%%%%%%%
%%%%%%%%%%%%%%%%%%%%%%%%%%%%%%%%%%%%%%%%%%%%%%%%%%%%%%%%%%%%
%%%%%%%%%%%%%%%%%%%%%%%%%%%%%%%%%%%%%%%%%%%%%%%%%%%%%%%%%%%%

\begin{document}
\entetedynamique

%%
%% Voici des options pour annoter les différentes versions de votre
%% mémoire. La commande \brouillon imprime, au bas de chacune des pages, la
%% date ainsi que l'heure de la dernière compilation de votre fichier.
%%
%%\brouillon
%%
%%
%% \version est la version de votre manuscrit
%%
\version{1}

%%------------------------------------------------- %
%%              pages i et ii                       %
%%------------------------------------------------- %

%%%
%%% Voici les variables à définir pour les deux premières pages de votre
%%% mémoire.
%%%

\title{Contemporary perspectives on the hyper-organ: conceiving, playing and writing for an augmented Casavant Frères pipe organ}

\author{Kjel Sidloski}

\copyrightyear{2024}

\department{Faculté de musique}

\date{\today} %Date du DÉPÔT INITIAL (ou du 2e dépôt s'il y a corrections majeures)

\sujet{musique}
\orientation{composition}%Ce champ est optionnel
%%
%% Voici les disciplines possibles (voir avec votre directeur):
%% \sujet{statistique},
%% \sujet{mathématiques}, \orientation{mathématiques appliquées},
%% \orientation{mathématiques fondamentales}
%% \orientation{mathématiques de l'ingénieur} et
%% \orientation{mathématiques appliquées}

\president{Nom du président du jury}

\directeur{Pierre Michaud}

\codirecteur{Caroline Traube}         % s'il y a lieu
%%\codirecteurs{Nom du 2e codirecteur}         % s'il y a lieu

\membrejury{Nom du membre de jury}

%%\examinateur{Nom de l'examinateur externe}   %obligatoire pour la these

%% \membresjury{Deuxième membre du jury}  % s'il y a lieu

%%  \plusmembresjury{Troisième membre du jury}    % s'il y a lieu

 % Cette option existe encore, mais elle n'a plus sa place
 % dans la page titre. L'utiliser seulement si le directeur
 % insiste...
%%\repdoyen{Nom du représentant du doyen} %(thèse seulement)

%%
%% Fin des variables à définir. La commande \maketitle créera votre
%% page titre.

\maketitle

 % Pour générer la deuxième page titre, il faut appeler à nouveau \maketitle
 % Cette page est obligatoire.
\maketitle

%%------------------------------------------------- %
%%              pages iii                           %
%%------------------------------------------------- %

\francais

\chapter*{Résumé}

...sommaire et mots clés en français...

%%------------------------------------------------- %
%%              pages iv                            %
%%------------------------------------------------- %

\anglais

\chapter*{Abstract}

This paper presents a body of work for the pipe organ exploring the emerging tradition of augmented instruments as pioneered by Tod Machover and others. 
The organ is being examined as an augmented instrument by a niche community including Lauren Redhead and the Orgelpark project, yet represents an underexploited resource of musical exploration, uniquely positioned as an embodied cultural artifact. 
The nature of an instrument that is embedded in its space poses the question of whether there is truly a distinction between the building housing the instrument and the instrument itself, yielding a situation in which augmenting the pipe organ can be seen akin to sonic architecture. 
The instrument examined in this work is the symphonic organ of l'église Saint-Édouard where the author is organist. 
Built in 1913 by Casavant Frères, this organ has a unique and eventful history. 
Being taken down from its original position in the west gallery, forgotten about and nearly abandonned, it was found by new administration nearly ten years later and reinstalled in the north transept where it sits today. 
I believe that the sound of this instrument tells the story of its heritage, and by studying its unique properties, we gain insight into the nuances of history. 
This project contains both an experimental and a creative component. 
The former includes a small study using Max modules and an online questionnaire to establish affective-poetic correlates between various organ registrations and semantic descriptors, as well as the creation of a sound synthesis module in python. 
The creative portion includes several compositions. 
The piece Élégies, inspired by the 10 Élégies de Duino (Rilke, 1986), serves as a set of studies to explore the capabilities of the hyper-organ, according to three modalities of mixed music: acoustic organ and synthetic organ, organ with processing, and organ with bed track. 
The church space is also considered in its entirety, invoking a spatial-narrative structure. 
Next, the project is brought into a collaborative context, with a piece for hyper-organ and viola. 
Throughout the creative process, attempts have been made to not only extend the sonic capacity of the instrument, but to integrate with it, basing for instance, the sound synthesis on various organ stops, which can then be mutated in ways which would be impossible on the original instrument, creating a dialectic of real and artifical sound environments.

%%------------------------------------------------- %

%%        page v --- Table de matieres              %
%%------------------------------------------------- %

 % Pour un mémoire en anglais, changer pour
 % \anglais. Noter qu'il faut une permission
 % pour écrire son mémoire en anglais.
\anglais
%%\francais
 % \cleardoublepage termine la page actuel et force TeX
 % a poussé les éléments flottant (fig., tables, etc.) sur
 % la page (normalement TeX les garde en suspend jusqu'à ce
 % qu'il trouve un endroit approprié). Avec l'option <twoside>,
 % la commande s'assure que la prochaine page de texte est sur
 % le recto, pour l'impression. On l'utilise ici
 % pour que TeX sache que la table des matières etc. soit
 % sur la page qui suit.
%% TABLE DES MATIÈRES
\cleardoublepage
\pdfbookmark[chapter]{\contentsname}{toc}  % Crée un bouton sur
                                           % la bar de navigation
\anglais
\tableofcontents
 % LISTE DES TABLES
\cleardoublepage
\anglais
\english
\phantomsection  % Crée une section invisible (utile pour les hyperliens)
\listoftables
 % LISTE DES FIGURES
\cleardoublepage
\phantomsection
%% Il est obligatoire, selon les directives de la FESP,
%% pour une thèse ou un mémoire d'avoir une liste des sigles et
%% des abréviations.  Si vous considérez que de telles listes ne seraient pas
%% pertinentes (si, par exemple, vous n'utilisez aucun sigle ou abré.), son
%% inclusion ou omission est laissé à votre discrétion.  En cas de doute,
%% parlez-en à votre directeur de recherche, le coadministrateur ou au/à la
%% bibliothécaire.
%%
%% Le gabarit inclut un exemple d'une liste « fait à la main ».  Il existe des outils
%% plus sophistiqués si vous devez inclure une multitude de sigles et abréviations.
%% Par exemple, le package <glossaries> peut faire des index élaborés.  Comme
%% son utilisation est technique, il n'y a pas d'exemple directement dans ce gabarit.
%% On invite les gens qui aurait à l'utiliser à lire la documentation officielle,
%% soit en allant sur https://www.ctan.org/, soit en tapant dans un terminal :
%%
%% texdoc glossaries
%%

\chapter*{List of acronyms and abbreviations}
 % Option de colonnes: definir \colun ou \coldeux
%%% Exemple
%%% \def\colun{\bf} % Première colonne en gras
%%% Pour numéroté les entrées, on peut faire
%%% \newcount\abbrlist
%%% \abbrlist=0
%%% \def\plusun{\global\advance\abbrlist by 1\relax}
%%% \def\colun{\plusun\the\abbrlist. }
%%\def\coldeux{\relax}
\begin{twocolumnlist}{.2\textwidth}{.7\textwidth}
\end{twocolumnlist}
%% L'environnement <threecolumnlist> existe aussi pour trois colonnes.

%%------------------------------------------------- %
%%              pages vi                            %
%%------------------------------------------------- %

\chapter*{Acknowledgements}

...remerciements...

 %
 % Fin des pages liminaires.  À partir d'ici, les
 % premières pages des chapitres ne doivent pas
 % être numérotées
 %

\NoChapterPageNumber
\cleardoublepage

%%%%%%%%%%%%%%%%%%%%%%%%%%%%%%%%%%%%%%%%%%%%%%%%%%%%%
%%                                                  %
%%   TEXTE DU MÉMOIRE :  introduction page 1,...    %
%%                                                  %
%%%%%%%%%%%%%%%%%%%%%%%%%%%%%%%%%%%%%%%%%%%%%%%%%%%%%

\chapter*{Introduction}

...introduction...

%%------------------------------------------------- %
%%                pages 1                           %
%%------------------------------------------------- %

\chapter{The Emergence and Relevance of Hyper-Instruments}

Hyper, or augmented instruments, situated at the confluence of traditional musical artistry and cutting-edge technological innovation, present a complex and diverse field of inquiry.
This chapter seeks to delve into this multifaceted world, highlighting the significance of hyper-instrument design and performancein contemporary music practices and the intricate process of developing and defining them within various musical contexts.

At the core of any musical instrument lies the interface between an artist's musical expression and the resultant sound in their environment.
Augmenting an instrument, therefore, becomes an exercise in expanding this interface.
This task is often far from straightforward, as musicians are typically engaged fully, utilizing their hands, mouth, and sometimes feet, to produce sound.
In such scenarios, where every faculty of mind and body is already employed, the question arises: where is there room for expansion?
Among the various approaches in interface design, are gestural mapping, sensitivity mapping through pressure sensors, and the integration of MIDI interfaces.
Each of these approaches has its unique challenges and opportunities, tailored to the specificities of the instrument being augmented.

The choice between convergence or divergence with the natural workings of the instrument is crucial.
For instance, augmenting a wind instrument involves navigating its intimate relationship with the performer's breathing and the air column.
Here, the lateral movements of the performer, which do not inherently affect pitch, provide an opportunity for introducing an independent dimension of expressivity and virtuosity.
Conversely, in an instrument like the piano, where pitch is intrinsically tied to the horizontal plane, any augmentation must consider this inherent constraint.
While this may limit the scope for independent controls, it ensures that any technological integration remains deeply rooted in the natural mechanics and traditions of piano playing.

This chapter aims to navigate the extensive landscape of hyper-instruments, examining methodologies and concepts that are shaping this evolving domain.
From the pioneering work of Tod Machover to adaptations in instruments like the hyper-cello, hyper-shakuhachi, hyper-piano, and hyper-flûte, we will explore how musicians and technologists are redefining musical expression.
Through this exploration, we seek to uncover the broader implications for music as a dynamic and evolving art form in the digital age.

\section{History and approaches}

\subsection{Tod Machover and the birth of the hyper-instrument}

La premi\`ere section du 1$^{\text{er}}$ chapitre.

\subsection{Levensons' hyper-cello}

\subsection{Beilharz' hyper-shakuhachi}

\subsection{McPhersons hyper-piano}

\subsection{Cléo Palacio-Quintin's hyper-flûte}

\section{Hyper-organ}

In the context of augmented pipe organs, these considerations manifest in unique and innovative ways.
Projects led by visionaries like Lauren Redhead and the Orgelpark initiative are exemplary in demonstrating how technology can reimagine the traditional pipe organ.
These initiatives not only showcase the potential of technological integration with historic instruments but also highlight the diverse methodologies and creative opportunities within the field of hyper-instruments.

\subsection{Lauren Redhead}

\subsection{Orgelpark}

\section{The question of coherence}

As we can see, augmented instrument design is far from a monolithic, and this inherent plurality is one of the most intriguing aspects of the field.
The term 'hyper-instrument' encompasses a broad spectrum of augmented musical tools, each with its unique characteristics and requirements.
Developing an intuitive interface for a hyper-instrument involves a deep understanding of the instrument's core essence and a thoughtful integration of technology that enhances without overshadowing its intrinsic qualities.
This diversity in approach underscores the difficulty in pinning down hyper-instruments as a singular practice.
Instead, they represent a constellation of practices, each illuminating different aspects of the intersection between modern and traditional technologies.

\chapter{Constructing an augmented interface for the pipe-organ}

\section{Context: L'église Saint-Édouard}

\section{Modalities}

\subsection{Synthesis (intentions of mimicry and mutation}

\subsubsection{Additive}

\subsubsection{Subtractive}

\subsubsection{Analysis approach}

\subsubsection{Setup/physical constraints}

\subsection{Live processing and effects}

\subsubsection{Issues of microphone capture and amplification in live setting}

\subsubsection{Delay}

\subsubsection{Distorsion}

\subsubsection{Reverb}

\subsection{Bed tracks and triggers}

\chapter{10 Élégies}

\section{Context and precedents}

\section{Compositional considerations}

\subsection{Constraint vs freedom}

\subsection{Artificial vs real}

\subsection{Spatiality and sonic architecture}

\subsection{Metaphors}

\subsection{History vs contemporainity}

\subsection{Memory/decay}

\subsection{Dialectic and fusion}

As seen through the various axes discussed--the voice vs the organ, constraint vs freedom, artificial vs real, history vs contemporainity, etc., dialectic thought is very important to my approach.
In my philosphy, the binary is the most fundamental building block of thought.
Even the most basic idea precludes a distinction--between self and other, or between this and that, for instance.
To me the entire world exists inside this divine contradiction, and art in its highest form is like the impossible resolution of these opposing forces.
As far as we know, nuclear fusion is the most immense creative force of energy production that exists.
The triggering of this fusion process is not trivial however, requiring a temperature of more than one-hundred million degrees fahrenheit.
Perhaps the fusion, or synthesis, of ideas in the realm of art is just such a creative force, requiring an equally massive force to hold them together.

\section{Tools}

\subsection{GrandOrgue}

\subsection{OrganLab}

\subsection{Open Music}

\subsection{Ableton Live}

\section{Movements}
\anglais
\subsection{1e Élégie}

\epigraph{\textit{Qui, si je criais, qui donc entendrait mon cri parmi les hiérarchies des Anges?}}{}

\subsubsection{Narrative context}
Rilke's Élégies begin not with a cry, but with the question of a cry.
A question seemingly addressed to no one.
A plea of solitude and desperation.
Perhaps this question is addressed to God, to the reader of his poem, or to himself.
In any case this question represents perhaps a sort of internal cry, representing the subjectivity of our protagonist's lament.
He goes on to evoke the level of solitude felt in his unknown setting, wondering if anyone is listening at all, calling into question the purpose of crying, or of communicating in any way.

From a sonic perspective, the issue of sound is immediately relevant, the cry evoking at once a sound source---the human voice---and the volume, and intensity of this sound.
The subjective nature of sound is also essential, with "qui donc entendrait" illustrating the importance of hearing and perception to the auditory experience.
This startling juxtaposition between the immense loudness of the subject's inner world, and the expansive quietude of the environment provides an immediate dialectic between inner and outer worlds.

The second part of the sentence describes the setting somewhat, though in obscure terms.
He depicts the subject as among the hierarchies of angels.
This brings another essential dialectic lense to Rilke's world: the spiritual and the human.
Here is a startling image.
We were quickly led to believe that we were alone, contemplating the futility of communication, when not only are we not alone, but we are found among a cacophonie of angels.
The reason that the angels might not hear the subject's plaintif cry is unclear. 
Maybe the angels are simply too far away.
Maybe they are making too much noise themselves to hear anything.
Or maybe they do not contain the capacity for subjective experience at all.
If we take this last case, this quickly paints an image of Rilke's angel, not as an anthropomorphized, man's best friend, but as a very different creature entirely.
The idea that they would hear nothing implies an austere, cold distance from the angels---the small human alone underneath a swarm of ethereal beings.

\subsubsection{Techniques}

Software:

One of the main advantages of a digital pipe-organ over an acoustic instrument, is the ability to use non-discrete pitches.
To represent the hierarchies of angels, I thought that it would be appropriate to make use of an extended glissando, starting in a lower register and moving upwards, signifying some kind of transcendental nature.
It might seem strange to represent a hierarchy with something non-discrete and essentially non-hierarchical, but the hierarchy is already present in the harmonic series.
The way that this glissando function works, is that over the course of a set amount of time (in this case two minutes), the frequency of each of the eight partials is incremented by one about ten times a second.
Because the frequency is being incremented, and not the midi equivalent, the harmonic series quickly becomes inharmonic.
For instance, if 100 and 200 are the first two partials, after an addition of 10, 110, and 210 are no longer multiples of eachother.
This process creates the sensation of an alien sort of movement, pulsating and morphing, yet not in a coherent, human way.    
\begin{figure}[H]
\begin{lstlisting}[language=Python]
glissC = [0 for i in range(8)]
def glissUp():
    global glissC
    for i in range(len(glissC)):
        glissC[i] == 0
    if glissC[0] < 600:
        stop1.setTrans(glissC)
        for i in range(len(glissC)):
            glissC[i] = glissC[i] + 2
    else:
        for i in range(len(glissC)):
            glissC[i] = 0
\end{lstlisting}
\caption{The glissUp function takes a python list and increments it until it reaches 600, this list is used to increment the first 8 partials of the synthesis module}
\end{figure}

Compositional:

For this movement, I wanted to represent the binary form of the poetic fragment with a juxtaposition of proportional notation and semi-metric notation.
The proportional notation is supposed to represent the angelic form.
Here again the choice is somewhat arbitrary.
One could easily make the case that metrical time is more hierarchical.
At the same time, metrical time has a long tail of history in western notation, and has for me a notion of human comprehensibility, whereas proportional notation is newer, with less historical associations.
Furthermore, proportional notation seems to give the sense of music frozen and static in time, which I associate with the timelessness of the ethereal.

\subsubsection{Composition Process}

This movement begins with a sung melodic fragment, which represents the principal melody of the piece, taken from a session of singing the poetry of Rilke in l'église Saint-Édouard in the fall of 2022.
The melody comes from the first line of the first elegie and has been modified only slightly from the original recording. 

\customincludegraphics{3-1-1_melody}{The main melody of the piece, based on the 19 syllables of the first line of Rilke's elegies.}

I wanted to then invoke a sonic interpretation of the cry.
Even though the cry is simply a hypothetical, and is not actually enacted in the poem, I wanted to represent the internal, subjective cry.
Of course there is no correct, objective way to represent subjectivity, but I've attempted a poetic approach.
To do this, I call upon a distinctly north american property of the pipe-organ: the crescendo pedal.
This pedal allows the player to quickly open or close many stops simply by flicking the foot.

One loses detailed control over the registration, yet gains access to a quick dynamic swell that surpasses the capacity of the normal expressive pedals.
I begin the piece with the crescendo pedal open about 60% of the way, with the acoustic and digital organs playing the same D minor chord.
At first, the acoustic instrument masks the digital sound completely, evoking the sens of solitude, but gradually the crescendo pedal closes and the acoustic instrument is masked, leaving the glissando as the focus.
The gradual widening of registral distance between the acoustic organ and the digital glissando evokes the sense of distance between the human and the angelic realms.

As far as the notation of the glissando, my original sketches included lines symbolizing the gesture of the sound and the many partials, and the ascent of the angels, but in the final manuscript, it felt like an unnecessary use of space on the page.
The organ-lab setting is desribed with the number one placed in a box, and upon placing the hand in the right position, the glissando will be handled by the software.

\customincludegraphics{3-1-2_gliss}{Original sketch of the glissandos of the first movement}

\customincludegraphics{3-1-3_system}{Final version of the first organ system, with the 1 in a box representing the first setting of Organ Lab, the sustain pedal used to maintain the glissando in the upper stave, and the crescendo pedal notated at the bottom.}

The choice of pitches in this movement come from a range of influences.
For one, one of the original inspirations of this piece was the Magnificat of Jehan Titelouze, and particular the sixth piece, Sexti Toni, which is based in F major.
For the first chord, I didn't want the brightness of F major, and decided to use the relative, D minor.
Not long after, on analysing the bells of l'église Saint-Édouard, I realized that they were also based in F major, which further jutified this decision.

Towards the end of the movement, the pipe organ slowly enters with ascending chords built around open fifths and fourths.
These mostly parallel, open harmonies for me evoke two influences.
On the one hand the organum tradition of Leonin and Perotin, and on the other hand, the open harmonies of Boards of Canada.
Ultimately, these chords rise like the synthesized glissando, representing the human who is caught under the ethereal.
Here the notation changes from proportional, to semi-metrical.
I use this word because the quarter note and half note of metrical notation are present, yet meter itself is not.

\customincludegraphics{3-1-4_system}{Rising chords based on open fifths and fourths}

\subsection{2e Élégie}

\epigraph{\textit{Tout Ange est terrible.
Et pourtant, malheur à moi!}}{}

\subsubsection{Narrative context}

In this movement, the haunting opening of Rilke's verse, "Tout Ange est terrible," forms the thematic crux.
The word 'terrible' presents a complex duality, oscillating between connotations of fear and awe and a sense of pathos and frailty.
This ambiguity is not just a linguistic puzzle but a profound emotional landscape that the music seeks to explore and express.
The latter part of the verse, "Et pourtant, malheur à moi!
pourtant je vous invoque," resonates as a solemn invocation, a plea before embarking on an arduous, soul-searching journey.
This movement attempts to capture the essence of this prayer, infused with both trepidation and a resigned yearning.

\subsubsection{Techniques}

Software:

The use of a gradual dynamic envelope is a pivotal element in this movement.
By enhancing the traditional bourdon sound with a brighter tone, and accentuating the fourth partial, a unique auditory experience is created.
This approach transcends the limitations of an acoustic pipe organ, illustrating the intersection of tradition and innovation.

\begin{figure}[H]
\begin{lstlisting}[language=Python]
def dynEnv():
    print('Enveloppe dynamique')
    stop1.setPart([1, 2, 3, 4, 4, 4, 0, 0])
    stop1.setMul([0.588, 0.338, 0.665, 0.773, 0.512, 0, 0, 0])
    stop1.setEnvAtt([0.285, 0.450, 0.327, 0.338, 0.385, 0.277, 0, 0])
    stop1.setEnvDec([0.02, 0.04, 0.085, 0.008, 0.008, 0.008, 0, 0])
    stop1.setEnvSus([0.446, 0.523, 0.404, 0.05, 0.05, 0.542, 0, 0])
    stop1.setEnvRel([0.1, 0.1, 0.1, 0.1, 0.1, 0.1, 0, 0])
    stop1.setNoiseAtt(0.081)
    stop1.setNoiseDec(0.146)
    stop1.setNoiseSus(0.7)
    stop1.setNoiseRel(0.1)
    stop1.setNoiseMul(3)
    stop1.setNoiseFiltQ(3)
    stop1.setSumMul(0)
\end{lstlisting}
\caption{The dynEnv function alters the dynamic envelope of both the harmonic sound and the noise sound, while tripling the fourth partial}
\end{figure}

Compositional:

The structure of the movement employs a binary form to parallel the contrasting themes of Rilke's verse.
Part A, representing the 'terrible angels,' is characterized by dense chromatic harmonies and a rich, plein orgue registration, employing proportional notation for depth.
In contrast, Part B adopts a simpler harmonic language, employing metric notation.
This section features an inverted version of the initial melody, serving as a cantus firmus.
Interestingly, all durations in this voice are equalized as whole notes, played with the pedals along with the bourdon and flute 8’.
This structure serves both as an improvisational guide and a compositional framework, offering the performer the liberty to improvise around the notated cantus firmus, colored in red, with an alternative pre-composed version available as a guideline.

\subsubsection{Composition Process}

The creation of this movement began with the creation of the sung melodic line.
I considered using new motivic material, but ultimately decided that it would be better to use a variation of the original melody.
This is partly to maintain coherence, and partly in reference to the Magnificat of Jehan Titelouze which uses a repetition of a sung melodic fragment as a formal element.
This thematic variation is now infused with additional chromaticism, reflecting a progression into complexity and uncertainty.

\customincludegraphics{3-2-1_melody}{A variation of the first sung melody, with more chromaticism}

The creation of Section A of the organ composition began with a dissonant harmonization of the main melody, making extensive use of an altered dominant sharp 7 chord.
This chord, a variation of the classic altered dominant chord in jazz, swaps the flat seven for a sharp 7, alongside the dominant 7 sharp 13 flat 9, and the minor major 7.
These harmonies, particularly resonant with the plein orgue registration, evoke an intense, almost ethereal quality.
The subtle voice leading within this dense texture aims to create movement and emotional depth.
Section B's composition process, focusing on the outer voices, was governed by the principles of traditional counterpoint.
This involved careful resolution of dissonances and maintaining a fluid musical dialogue between the voices.
The movement, starting from a melodic D minor perspective, culminates in a half cadence to the A minor dominant, signifying a transition yet an unfinished journey.

\customincludegraphics{3-2-2_system}{System showing the progression from dense chromatic harmonies in proportional notation to metrical notation with outervoices braiding an inner cantus firmus}

\subsubsection{Linking}

The movement concludes with an auditory link to the next – a sound sample of footsteps echoing in the basement of the church, leading into the sprinkler room.
The mundane, yet eerily resonant sounds of the church's underbelly, including the hum of compressors, provide a sensory bridge to the third movement.
This blending of the sacred with the profane mirrors the thematic journey of the music, from the divine to the earthly, the celestial to the subterranean.

\subsection{3e Élégie}

\epigraph{\textit{Chanter l'Amante est une chose. C'en est une autre, hélas! de chanter cet occulte Fleuve-Dieu du sang.}}{}

\subsubsection{Narrative context}

I interpret this poetic fragment as a reflection on the difficulty of walking the path ahead--contemplating the dichotomy between the relative simplicity of singing of love, versus the profound challenge of vocalizing the sacred, yet enigmatic, 'God river of blood.' 
This highlights a dialectic between thought and action, and the contrast between superficial perception and embodied reality.
This narrative conflict sets the tone for the movement's musical exploration.
Musically, I interpret "chanter", in a few ways, on the one hand, as the voix humaine stop on the récit, with tremolo, which imitates the sung human voice.
"Chanter l'Amante" is represented by a simple song-like melody, and "Chanter cet occulte Fleuve-Dieu du sang" is interpreted as the compositional and timbral disintegration of the movement.

\subsubsection{Techniques}

Software:

I thought that it would be appropriate to emulate the tremolo of the voix humaine stop with frequency modulation of a voix humaine emulation in OrganLab.
With this idea, it seemed like a logical step to increase this frequency modulation in speed, while changing the index and ratio, such that the natural trembling of the voice, passes through a comic, exaggerated warble, gradually losing the sense of sung tremolo altogether, giving way to an inharmonic, bell-like sound.
This is all done by interpolating with SigTo() objects, and the entire process takes place over 180 seconds.
The one part that is not interpolated, due to limitations of ADSR objects in Pyo, is the envelope, which simply changes suddenly at the end of the 180 seconds.

\begin{figure}[H]
\begin{lstlisting}[language=Python]
bellCall1 = None
bellCall2 = None
bellCall3 = None
bellCall4 = None

def bell():
    global bellCall1, bellCall2, bellCall3, bellCall4 
    bellCall1 = CallAfter(stop1.setEnvAtt, time=180, arg=(.001, .001, .001, .001, 0.001, 0.001, 0.0001, 0.0006, 0.0007, 0.0005, 0.0006, 0.0003, 0.0005, 0.0003, 0.0006, 0.0005, 0.0004, 0.0002, 0.0001, 0.0001)).play()
    bellCall2 = CallAfter(stop1.setEnvDec, time=180, arg=(1.3, .05, .02, 0, 0, 0.04, .004, 0.04, .04, 0.04, .04, 0.04, .04, 0.04, .04, 0.04, .04, 0.04, .04, 0.04)).play()
    bellCall3 = CallAfter(stop1.setEnvSus, time=180, arg=(.4, .1, .02, .01, .01, 0.01, .01, 0.01, .01, 0.01, .01, 0.01, .01, 0.01, .01, 0.01, .01, 0.01, .002, 0.002)).play()
    bellCall4 = CallAfter(stop1.setEnvRel, time=180, arg=(2, 0.1, 0.1, .01, .03, 0.4, .04, 0.04, .04, 0.04, .04, 0.04, .04, 0.04, .04, 0.4, .04, 0.04, .04, 0.4)).play()
    setInterpol(180)
    stop1.setRamp(180)
    stop1.setMul([1, 0.01, 0.1, 0.01, 0.07, 0, 0.02, 0, 0.01, 0, 0.003, 0, 0.003, 0, 0.001, 0, 0.001, 0, 0.001, 0])
    stop1.setRatio(0.43982735)
    stop1.setIndex(4)
    stop1.setNoiseAtt(0.001)
    stop1.setNoiseDec(0.1)
    stop1.setNoiseSus(0.01)
    stop1.setNoiseRel(0.1)    
    stop1.setNoiseMul(0.9)
    stop1.setNoiseFiltQ(4)
    stop1.setPartScRat(1.02)
    print(bell)
\end{lstlisting}
\caption{This function morphes the ratio and index of frequency of modulation, as well as a very slight exponential expansion of the harmonic series on the attack, which creates an additionally inharmonic attack. The envelope is altered after 180 seconds through the bellCall function calls}
\end{figure}

Interpolation in general was a major goal of this project, and one of my initial ideas was the transition between harmonic organ sounds, and inharmonic and/or chaotic sounds.
This process is semantically justified through the passage from the simplicity of singing of love, versus the reality of singing the God river of blood.
It is as though the simple dream is interrupted by a descent into a more complextimbral world.

Compositional:

The movement adopts a song form to represent 'singing,' starting with an 12 bar melody as the A section.(Grove reference)
Instead of following a traditional AABA form, the A section is repeated incessantly in a variation form.
These variations expand exponentially at a rate of 5/4, symbolizing a narrative disruption akin to a rupture in the fabric of time.
During these variations, the left hand chords and rhythms are constrained, while the right hand is free to improvise. 
A composed version is offered as a suggestion, or as an alternative to improvisation.
To parallel the software techniques, frequency modulation is also applied at the notational level.
A patch in OpenMusic uses the A section melody as the carrier and the bass line as the modulator to generate chords.
Selected sonorities from this process are integrated into the variations, enhancing the movement's thematic and textural depth.

\subsubsection{Composition Process}

The movement begins with a sung fragment, introducing a lilting melody, rising by thirds and spelling out an e minor triad.
The organ then enters with a similar rising motive, harmonized with perfect fifths below the melody.
Throughout this process the tonal center is tenous, first suggesting B minor in the voice, then A minor in the organ, with a thin, hollow texture of 8' and 2' flutes, before pivoting towards D minor.
This sets the stage for the entrance of the main motive of this movement, a lamenting descending line, beginning on the third scale degree, in this case F, and moving down towards the fifth scale degree, C in this case.
This descending line is in stark contrast with the rising motive heard earlier.
Both the tonal and motivic contrast between the opening sung melody, and the following part create a dissonance, not in a harmonic sense, but in a temporal, or symbolic sense.
The relation between the sung material, and the organ material is not yet clear.
Various alternatives to the lilting rising motive were tried for the introduction.
The original intention was to write something that felt like a classic song introduction in the tin pan alley tradition.
In the end, none of my more elaborate introductions seemed as appealing as the stark simplicity of the original rising motive.

\customincludegraphics{3-3-1_system}{Sung melodic fragment introducing lilting ascending motive}

\customincludegraphics{3-3-2_system}{Organ introduction, referencing the rising motive, but leaning towards A minor instead of B minor, which then leads into the main, lamenting melody at the end of the system.}

As for the composition of the descending motive, the first four measures came to me around the same time as the rising motive, around september, 2022, when I first started as organist at l'église Saint-Édouard and started my masters program at l'Université de Montréal.
Throughout the following months, I found myself adding on pieces to this melody, and finding new variations.
Once I decided to move in the direction of song form, I tried to crystallize the melody into a version that would fit within the bounds of an 8, 12 or 16 bar section I did this by taking all my motivic fragments and shuffling them around in different orders until I found a version that felt like it had a logical progression.
In the end, the idea seemed to naturally solidify into 12 bars.

At first, the continuation of this movement was supposed to simply dissolve into a textural mass, as the sound of the voix humaine undergoes it's painful transition into a bell.
This was the working idea until I realized that variation form, a form that I was looking to incorporate into the piece at some point, would be particularily appropriate for song form, as there is a great jazz tradition of playing variations on traditional songs.
In order to facilitate these variations, and to reference the jazz tradition, I decided to find jazz chords that could harmonize the original melody.
The original accompaniment is with an octave ostinato, and didn't lend itself well to variations in the jazz style, leading to an alternative harmonisation.

\customincludegraphics{3-3-3_melody}{Descending "lament" motive, with jazz changes annotated above}

At first, the continuation of this movement was supposed to simply dissolve into a textural mass, as the sound of the voix humaine undergoes it's painful transition into a bell.
This was the working idea until I realized that variation form, a form that I was looking to incorporate into the piece at some point, would be particularily appropriate for song form, as there is a great jazz tradition of playing variations on traditional songs.
At first, the continuation of this movement was supposed to simply dissolve into a textural mass, as the sound of the voix humaine undergoes it's painful transition into a bell.
This was the working idea until I realized that variation form, a form that I was looking to incorporate into the piece at some point, would be particularily appropriate for song form, as there is a great jazz tradition of playing variations on traditional songs.
The obvious and simple choice would have been simply to simply repeat the twelve bars over and over again, but I felt that this simple repetition wasn't necessarily in line with the subject matter, devolving into the God river of blood.
I was also hearing continuations in 3/8, and wanted to find a way to alter the meter, while keeping in line with the variations idea.
It then came to me to incorporate exponential variations.
The idea is that each variation is either dilated, or expanded by a certain proportion.
In this case, I chose the proportion of 5/4, finding that a proportion of 2/1 wouldn't give me the metric variety I was looking for, and 3/2 would be too long.
5/4 seemed to give me the right amount of variety without being overly long.
In order to verify this, I drafted a patch in OpenMusic that calculates the total time of a given number of measures with a given time signature at a given tempo.
With 12 measures at a tempo of quarter note equals 64 beats per minute, I found that with four variations, the movement would take about 4.3 minutes, not including the introductory material.
This seemed like a reasonable length, and gave me the time signatures of 5/4, 25/16, and 125/64.
Obviously, these latter two are all but unreadable to mere mortals, and required some clever compartmentalization to yield a musical, communicative notation.
I decided to break up 25/15 into three measures of 3/8 plus one measure of 7/16.
The 125/64 was more problematic.
After many attempts, including a systematic attempt to find every possible decomposition of the fraction, using a site similar to this one: https://www.mathcelebrity.com/decompose-fraction.php?num=125%2F64&pl=Decompose, only to find that all possible combinations would need a measure with a denominator of 64.
Using 64th notes as the rythmic pulse is not particularily practical, so my first instinct was to round to the 32nd note, but I then considered the possibility of incorporating a metric modulation.
Through some trial and error, I found that making five eighth notes equal to the new quarter note would allow me to use a measure of 2/4 plus a measure of 9/32 to equal my 125/64 beast.
9/32 is still not a particularily common time signature, but the grouping of three 32nd notes in compound time makes it more tenable.

\customincludegraphics{3-3-4_system}{First variation, in 5/4, with constrained elements notated in red, and free, optional element notated with small noteheads}

\subsubsection{Linking}

As the movement concludes, the sound of the bells gradually fades, transitioning into the sound of singing glasses.
This transition not only serves as a bridge to the next movement but also symbolizes the continual evolution of the thematic elements–-from the tangible to the ethereal, from the known to the unknown.

\subsection{4e Élégie}

\epigraph{\textit{Vous, Arbres de la Vie, oh! quand donc hivernaux? Nous n'allons pas à l'unisson.}}{}

\subsubsection{Narrative context}

On the one hand, this poetic fragment speaks to the trees of life.
The object that these trees symbolize is not clear.
Are they the angels, the readers of the poem, or perhaps life itself?
In any case, the poem then places the trees in the inevitability of the impending winter.
This winter can be read as an intimation of death, or at least a dormant state.
The musical term \textit{'unisson'} and its deliberate negation in the next sentence express a sense of divergence or disharmony, which becomes a central theme in the musical interpretation.

From a musical standpoint, I wanted to set the listener in the winter, which for me entails a sense of space and vastness, of stillness.
To evoke this environment, I thought it would be appropriate to use sounds with slow attacks and decays, to create a sense of slow evolutive timbre.
As I had been working on the writing of Jardin de Givre, which made extensive use of singing wine glasses, and I had high quality chromatic samples of these wine glasses on hand, I decided to create a sampler out of these recordings, so that I could play the wine glasses on my midi keyboard.
This also presents a pun, as the french word for ice is glace, which is a homophone with glass, the material of the wine glasses, evoking the ice covered ground of the desolate winter landscape.
As far as the second part of the poetic fragment, I decided that this would be a good moment to bring back the glissando motive from the first movement, this time moving in contrary motion.

\subsubsection{Techniques}

Software:

In this movement I make use of sampling for the first time, in two senses.
In one sense, with a sampler VST in Ableton Live which uses some recordings of singing wine glasses recorded in my apartment.
This sampler is driven by my midi keyboard and can be played chromatically.
On the other hand, I have a longer sample that is not played on the keyboard, but is simply triggered, acting as a bed track, to support the finale of the movement, creating additional tension and filling the space.
Lastly, in Pyo, I decided to revisit the glissando motive from the first movement, this time rather than moving all the partials in the same direction, moving the even partials higher in pitch, and the odd partials lower in pitch, creating a kind of contrary motion, while at the same time exploding the harmonic series.
This process refers to the text "Nous n'allons pas à l'unisson", creating a sonic divergence in opposing directions in pitch space.

\begin{figure}[H]
\begin{lstlisting}[language=Python]
def glissCont():
    global glissC
    for i in range(len(glissC)):
        glissC[i] == 0
    if glissC[0] < 600:
        stop1.setTrans(glissC)
        for i in range(len(glissC)):
            if i % 2 == 0:
                glissC[i] = glissC[i] + 0.4
            else:
                glissC[i] = glissC[i] + -0.4
    else:
        for i in range(len(glissC)):
            glissC[i] = 0
    print("0", glissC[0])
    print("1", glissC[1])
\end{lstlisting}
\caption{The function glissCont defines a glissando where each even partial is gradually incremented, whereas each odd partial is gradually decremented. This function is later called by a Pattern object 10 times a second}
\end{figure}

Compositional :

Several compositional constraints and techniques are explored in this movement.
On the one hand, the application of an isorhythm as cantus-firmus, which is placed in the bass part but is played with the right hand on the midi keyboard.
The right hand is very much not accustomed to playing the bass part, and so this represents a sort of étude.
Seeing as the keyboard can't easily be switched from one side to another, and playing it with the left hand would make it impossible to play the acoustic organ with the right hand, this provides a unique limitation and possibility for new cognitive and proprioceptive pathways.


I also wanted to develop the ascending by thirds motive, and decided to use it as an ostinato that begins slowly, later becoming a fast arppegiation, which in turn becomes the accompaniment for a variation of the melody of lamentation.
This provides a first attempt of the synthesis of these two motives, so central to the composition as a whole.
The arpeggiations also form, both visually in the score, and sonically, a set of arches.
These arches are symbolic of the physical arches of l'église Saint-Édouard and represent the broader iconography of the church.

\subsubsection{Composition Process}

This movement begins with an evocation of winter's vast stillness.
Long, monophonic singing glass samples with gradual attack and decay, unachievable on a traditional pipe organ, form the foundation.
These melodies are particularily centered on different combinations of the minor third, referencing both the initial melody and the ascending motif from the third movement.
These minor thirds are explored and combined with other intervals to form a fairly free and wandering section with ambigous tonality.

\customincludegraphics{3-4-1_system}{Free monophonic section, using singing bell samples with long attacks and decays.}

This melody uses proportional notation, to add to the metric freedom of the evolutive sounds. Later, a meter of 4/4 is introduced, and as the movement progresses, the intial long melody is truncated into a chroma of 12 notes, set against a talea of four rhythms: a dotted half note, a half note, a dotted half note rest, and a whole note.
This cycle repeats twice, leading to a shift where the right hand retakes the treble pitches, and the bass moves to the pedals.
The introduction of an ascending triad motive ostinato in C minor follows, with an audio file triggered to reinforce the ostinato, pitch-shifted an octave higher for a brighter sound.
However, just as C minor becomes established, the movement shifts to G minor.
A low G in the bombarde from the bed track cues rapid arpeggiations in the left hand, while the right hand introduces a descending melody reminiscent of the lamentation motif from the third movement.
This section, with its faster left-hand figurations and the bed track's added tension, can be likened to a winter storm, symbolizing chaos and intensity.

The crescendo pedal is used strategically to create large-scale dynamic waves, swelling and fading over approximately four-measure intervals.
In the final phase, the MIDI keyboard transitions from singing wine glasses to glissandi in contrary motion, culminating in a single sustained chord that fades into an ambient mass.

\customincludegraphics{3-4-3_system}{The introduction of the arch motive, symbolizing the staggering arches of l'église Saint-Édouard}

\subsubsection{Linking}

The transition from the third to the fourth movement is seamless, moving from bell-like sounds to singing glass sounds.
This continuity, however, necessitates a deviation from the established structure of singing before each Elegie.
To address this, the melody of the fourth Elegie is placed after the fourth movement, followed immediately by the melody of the fifth Elegie.
This arrangement, while a stark contrast, serves a specific purpose: the first melody acts as an echo of the preceding movement, while the second heralds the beginning of something new.
The common thread between these contrasting elements is the shared instrument of the voice, providing a thematic and auditory link.

\subsection{5e Élégie}

\subsubsection{Narrative context}

\subsubsection{Techniques}

\subsubsection{Composition Process}

\subsubsection{Linking}

\subsection{6e Élégie}

\subsubsection{Narrative context}

\subsubsection{Techniques}

\subsubsection{Composition Process}

\subsubsection{Linking}

\subsection{7e Élégie}

\subsubsection{Narrative context}

\subsubsection{Techniques}

\subsubsection{Composition Process}

\subsubsection{Linking}

\subsection{8e Élégie}

\subsubsection{Narrative context}

\subsubsection{Techniques}

\subsubsection{Composition Process}

\subsubsection{Linking}

\subsection{9e Élégie}

\subsubsection{Narrative context}

\subsubsection{Techniques}

\subsubsection{Composition Process}

\subsubsection{Linking}

\subsection{10e Élégie}

\subsubsection{Narrative context}

\subsubsection{Techniques}

\subsubsection{Composition Process}

\subsubsection{Linking}

\subsubsection{Linking}

%%--------------%
%%     index    %
%%--------------%

%% S'il y a lieu, décommenter la ligne pour mettre votre index

%%\printindex

%%------------------------------------------------- %
%%         références --- bibliographie             %
%%------------------------------------------------- %
  % Enlever les commentaires de la prochaine commande si vous préférez que le
  % chapitre s'appelle « Références » plutôt que « Bibliographie » (au choix selon le contexte).
%%\let\bibname=\refname

%% Lorsque vous serez prêt à faire afficher votre bibliographie
%% et vos références, enlevez les commandaires des commandes suivantes
%% et donnez le nom de votre fichier .bib à la commande \bibliography{..}
%% (consultez l'exemple au besoin).  Vous pouvez utiliser le style de votre
%% choix.
\bibliographystyle{plain}     % Le style de la bibliographie. Notons que
                                        % les extensions ne sont pas données pour ces deux fichiers.
%%\def\bibname{R\'ef\'erences bibliographiques} % Nom obligatoire de la section des références.
                                              % On utilise \'e car le é cause des problèmes
                                              % dans la table des matière
%% ENGLISH
\def\bibname{References}
\bibliography{ref}     % La base de données contenant des entrées bibliographiques.
                                    % Seules celles référencées dans le texte seront ajoutées
                                    % \`a la bibliographie.

%%------------------------------------------------- %
%%                  Annexe A                        %
%%------------------------------------------------- %

\appendix
\chapter{Le titre}

\section{Section un de l'Annexe A}

...texte...

\chapter{Les différentes parties et leur ordre d'apparition}

J'ajoute ici les différentes parties d'un mémoire ou d'une thèse ainsi
que leur ordre d'apparition tel que décrit dans le guide de
présentation des mémoires et des thèses de la Faculté des études
supérieures.  Pour plus d'information, consultez le guide sur le site
web de la facutlé (www.fes.umontreal.ca).

\newcount\colnum
\colnum=1
\def\i{\number\colnum. \global\advance\colnum by 1\ignorespaces}
\begin{table}[p]
  \begin{center}
    \begin{tabular}{|l|l|r|}\hline
       \noindent\hfil
         \textbf{\strut Ordre des éléments constitutifs du mémoire ou de la thèse}
         \hfil\span\omit\span\omit\\\hline % \span\omit pour couvrir plus d'une
                                           % case sans utiliser le package multirow ou autre
      \i &  La page de titre & obligatoire\\\hline
      \i &  La page d'identification des membres du jury & obligatoire\\\hline
      \i &  Le résumé en français et les mots clés français\kern3em& obligatoires\\\hline
      \i &  Le résumé en anglais et les mots clés anglais & obligatoires\\\hline
      \i &  Le résumé dans une autre langue que l'anglais & obligatoire \\
         &  ou le français (si le document est écrit dans &\\
         &  une autre langue que l'anglais ou le français)&\\\hline
      \i &  Le résumé de vulgarisation& facultatif\\\hline
      \i &  La table des matières, la liste des tableaux,& obligatoires\\
         &   la liste des figures ou autre &\\\hline
      \i &  La liste des sigles et des abréviations& obligatoire\\\hline
      \i &  La dédicace& facultative\\\hline
      \i &  Les remerciements & facultatifs\\\hline
      \i &  L'avant-propos & facultatif\\\hline
      \i &  Le corps de l'ouvrage& obligatoire\\\hline
      \i &  Les index& facultatif\\\hline
      \i &  Les références bibliographiques & obligatoires\\\hline
      \i &  Les annexes & facultatifs\\\hline
      \i &  Les documents spéciaux & facultatifs\\\hline
    \end{tabular}
  \end{center}
\end{table}

\end{document}



\endinput
%%
%% End of file `gabaritmem.tex'.
